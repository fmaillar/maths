%pour résoudre le problème "No room for a new dimen" et on le place en premier
_{\}usepackage{etex}

%packages fondamentaux
\usepackage[utf8]{inputenc}
\usepackage[T1]{fontenc}
\usepackage[francais]{babel}

%pour les symboles mathématiques
\usepackage{amsthm}
\usepackage{amsmath}
\usepackage{amssymb}
\usepackage{amsfonts}
\usepackage{mathrsfs}
\usepackage{latexsym}

%Pour faire des maths en francais
\usepackage{frmath}

%Pour utiliser les tableaux
\usepackage{array}

%fonte vectorielle
\usepackage{lmodern}

%paquet pour les liens hypertextes
\usepackage[colorlinks=true, linkcolor=black, urlcolor=black]{hyperref}

%pour les marges
\usepackage{geometry}
%\geometry{margin=2cm}
\geometry{vmargin=2cm}

%pour enlever la numérotation des pages
\pagestyle{empty}

%Pour écrire de la physique avec les unités
%\usepackage{siunitx} 

%Pour inclure des images avec \includegraphics[scale=]{}
\usepackage{graphics}

%Pour inclure des fichiers .ps ou .eps
\usepackage{pstricks-add}

%Pour dessiner avec Tikz
\usepackage{tikz, pgf}
\usetikzlibrary{arrows,calc,backgrounds}

% pour le symbole \danger
\usepackage{fourier-orns} 

%pour utiliser les couleurs
\usepackage{color}

%pour utiliser "draft" (brouillon)
%\usepackage{draftcopy}

%pour renvoyer les notes à la fin
\usepackage{endnotes}

%pour les encadrements et les boites
\usepackage{fancybox}

%pour les boucles de programmation en \LaTeX
\usepackage{ifthen}

%Pour le placement de flottants avec FloatBarrier
\usepackage{placeins}
%\makeatletter
%	\renewcommand\section{\@startsection
%	{section}{2}{0mm}
%	{-\baselineskip}{0.5\baselineskip}
%	{\FloatBarrier\normalfont\Large\bfseries}}
%\makeatother


%Calculs arithmétique
%\usepackage{calc}

%Lettrines
\usepackage{lettrine}
%pour le symbole double crochet \rrbracket \llbracket
\usepackage{stmaryrd}
%pour les minitoc{}
\usepackage{minitoc{}}
\mtcselectlanguage{french}

%pour les tableaux de variations
\usepackage{variations}

%% entêtes et pieds de pages
%Entêtes et pieds de pages
\usepackage{fancyhdr}
\pagestyle{fancy}

\fancyhead{}
\fancyfoot{}

\renewcommand{\subsectionmark}[1]{%
  \ifsubsectioninheader
    \def\subsectiontitle{: #1}%
  \else
    \def\subsectiontitle{}%
  \fi}
\newif\ifsubsectioninheader
\def\subsectiontitle{}

%\setlength{\headheight}{15.36pt}

\theoremstyle{plain} \newtheorem*{theo}{Théorème}
\newtheorem*{lemme}{Lemme}
\theoremstyle{definition} \newtheorem*{defdef}{Définition}
\newtheorem*{prop}{Proposition}
\newtheorem*{rappel}{Rappel}
\newtheorem*{proprietes}{Propriétés}
\theoremstyle{plain} \newtheorem*{cor}{Corollaire}

%\linespread{1.1}

%%%%%%%%%%%%%%%%%%%%%%%% COMMANDES PERSOS %%%%%%%%%%%%%%%
\newcommand{\fonction}[5]{#1\colon \left\{\begin{array}{ccc}
#2 & \longrightarrow &#3\\
#4 & \longmapsto & #5
\end{array}
\right.
}

\newcommand{\Rplusetoile}{\left]0,+\infty\right[}
\newcommand{\N}{\mathbb{N}}
\newcommand{\R}{\mathbb{R}}
\newcommand{\Rbar}{\bar{\R}}
\newcommand{\K}{\mathbb{K}}
\newcommand{\Z}{\mathbb{Z}}
\newcommand{\C}{\mathbb{C}}
\newcommand{\Q}{\mathbb{Q}}
\newcommand{\A}{\mathcal{A}}
\renewcommand{\P}{\mathcal{P}}
\newcommand{\I}{\mathcal{I}}
\newcommand{\Rep}{\mathcal{R}}
\newcommand{\Dr}{\mathcal{D}}
\newcommand{\Def}[1]{\Dr_{#1}}
\newcommand{\U}{\mathbb{U}}
\newcommand{\Part}{\mathfrak{P}}
\renewcommand{\H}{\mathcal{H}}
\newcommand{\E}{\mathcal{E}}
\newcommand{\classe}[1]{\mathit{C}^#1}
\newcommand{\con}[1]{\mathcal{C}_#1}
\newcommand{\courbe}{\mathcal{C}}
\newcommand{\vi}{\vect{i}}
\newcommand{\vj}{\vect{j}}
\newcommand{\vk}{\vect{k}}
\newcommand{\vu}{\vect{u}}
\newcommand{\vv}{\vect{v}}
\newcommand{\vw}{\vect{w}}
\newcommand{\vz}{\vect{z}}
\newcommand{\vn}{\vect{n}}
\newcommand{\va}{\vect{a}}
\newcommand{\vb}{\vect{b}}
\newcommand{\vc}{\vect{c}}
\newcommand{\rond}{(O,\vi,\vj)}
\newcommand{\bond}{(\vi,\vj)}
\newcommand{\rondtrois}{(O,\vi,\vj,\vk)}
\newcommand{\bondtrois}{(\vi,\vj,\vk)}
\newcommand{\rondtroisu}{(\Omega,\vu,\vv,\vw)}
\newcommand{\Ell}{\mathcal{E}}

\DeclareMathOperator{\sgn}{signe}% fonction signe
\DeclareMathOperator{\e}{e}% exponentielle
\DeclareMathOperator{\argch}{argcosh}% argument cosinus hyperbolique
\DeclareMathOperator{\argsh}{argsinh}% argument sinus hyperbolique
\DeclareMathOperator{\argth}{argtanh}% argument tangente hyperbolique
\DeclareMathOperator{\cotan}{cotan}% cotangente
\DeclareMathOperator{\sinc}{sinc}% sinus cardinal
\DeclareMathOperator{\Arg}{Arg}% argument principal
\DeclareMathOperator{\ii}{i}% unité imaginaire
\DeclareMathOperator{\Ent}{E}% fonction partie entière
\DeclareMathOperator{\Id}{Id}% fonction identité
\DeclareMathOperator{\Det}{Det}% déterminant
\DeclareMathOperator{\Inf}{inf}% le plus grand des minorants
\DeclareMathOperator{\Card}{Card}% le cardinal d'un ensemble

\newcommand{\cercle}[2]{\mathcal{C}(#1,#2)}% cercle
\newcommand{\congru}[3]{#1 \equiv #2~\left[#3 \right]} %congru
\newcommand{\expc}[2]{#1 \e^{#2 \ii}}% exp complexe
\renewcommand{\bar}{\overline}
\newcommand{\Uv}{\vect{u}}
\newcommand{\V}{\vect{v}}
\newcommand{\W}{\vect{w}}
%% pour les dérivées
\newcommand{\derived}[2]{\dfrac{\D #1}{\D #2}}
\newcommand{\deriveds}[2]{\dfrac{\D^2 #1}{\D {#2}^2}}
\newcommand{\derivep}[2]{\dfrac{\dr #1}{\dr #2}}
\newcommand{\deriveps}[2]{\dfrac{\dr^2 #1}{\dr {#2}^2}}
\newcommand{\derivepc}[3]{\dfrac{\dr^2 #1}{\dr {#2} \dr {#3}}}
%%
\newcommand\Hrule{\noindent \rule[0mm]{\linewidth}{0.5pt}}

%% Diagramme de Venn
\newcommand{\En}{(-4,-1) rectangle (4,4)}
\newcommand{\An}{(0,0) ++(135:2) circle (2)}
\newcommand{\Bn}{(0,0) ++(45:2) circle (2)}
\newcommand{\AnB}{(0,0) arc (-45:45:2) arc (135:225:2)}
\newcommand{\AuB}{(0,0) arc(-135:135:2) arc(45:315:2)}
\newcommand{\AmB}{(0,0) arc (225:135:2) arc (45:315:2)}
\newcommand{\BmA}{(0,0) arc(-135:135:2) arc(45:-45:2)}



